%******************************************************
%*    LaTex-Vorlage f�r wissenschaftliche Arbeiten    *
%*    --------------------------------------------    *
%* 		  Autor: Jan Brennenstuhl (@jbspeakr)           *
%*						 www.funkblocka.de                      *
%* 		Version: 0.1								                    *
%* 		  Datum: 07.09.2012							                *
%*     Lizenz: CC BY-SA 3.0                           *
%******************************************************
%*  Anmerkung: Unter Verwendung der Paket-            *
%*             Empfehlungen von Georg Verweyen und    *
%*             Teilen der Vorlage von Tino Weinkauf   *
%******************************************************
\documentclass[
  paper=A4, 			% Stellt auf A4-Papier ein
  pagesize, 			% Diese Option reicht die Papiergr��e an alle Ausgabeformate weiter
  DIV=calc, 			% F�r einen guten Satzspiegel
  headings=small,	% F�r etwas kleinere �berschriften
  ngerman,  			% Neue Rechtschreibung (Silbentrennung)
  12pt, 					% Schriftgr��e
	listof=totoc, 
  bibliography=totoc, 
  index=totoc, 
	BCOR5mm, 				% Formatregeln f�r Bindung
	]{scrbook}
% ---------------------------------------------------------------
% Unverzichtbare Pakte
\usepackage[T1]{fontenc}			% f�r Texte mit Umlauten und/oder Akzenten 
\usepackage[latin1]{inputenc}	% Silbentrennung und Eingabe von Umlauten
\usepackage{fixltx2e}  				% Verbessert einige Kernkompetenzen von LaTeX2e
\usepackage[ngerman]{babel} 	% Deutsch als Hauptsprache
\usepackage{natbib} 
\usepackage{graphicx} 				% Einf�gen von Grafiken (.png)
\usepackage[hyphens]{url} 		% URL's sch�ner formatieren
\usepackage{hyperref}         % Hyperlinks im PDF-Dokument
%\usepackage{pdfpages} 				% Einf�gen von PDF-Dateien
% ---------------------------------------------------------------
% Typografisch empfehlenswerte Pakete
\usepackage{% 
  ellipsis, % Korrigiert den Wei�raum um Auslassungspunkte
  ragged2e, % Erm�glicht Flattersatz mit Silbentrennung
 marginnote,% F�r bessere Randnotizen mit \marginnote statt % \marginline
}
\usepackage[tracking=true]{microtype}%
\DeclareMicrotypeSet*[tracking]{my}% 
  { font = */*/*/sc/* }% 
\SetTracking{ encoding = *, shape = sc }{ 45 }% alle Passagen in Kapit�lchen automatisch leicht gesperrt.
% ---------------------------------------------------------------
% Verschiedene Schriften
\usepackage{%
  lmodern, % A) Latin Modern Fonts sind die Nachfolger von Computer
            % Modern, den LaTeX-Standardfonts
%  hfoldsty % B) Diese Schrift stellt alle Ziffern, au�er
            % im Mathemodus, auf Minuskel- oder Medi�val-Ziffern um.
            % Wenn Ihre pdfs unscharf aussehen installieren Sie bitte
            % die cm-super-Fonts (Type1-Fonts).
% charter   % C) Diese Zeile l�dt die Charter als Schriftart
}
% ---------------------------------------------------------------
% Glossar-Anpassungen

\usepackage[style=long,border=none,header=plain,cols=3]{glossary}
	\makeglossary
	\renewcommand{\entryname}{K�rzel}
	\renewcommand{\descriptionname}{Beschreibung}
\usepackage{multibib}
	\newcites{buch}{Literaturverzeichnis}
	\newcites{web}{Web-Quellenverzeichnis}
% ---------------------------------------------------------------
% Einstellungen
\pdfminorversion=6
\input{hyphenation}
\graphicspath{{./gfx/}}
% ---------------------------------------------------------------
\begin{document}
%******************************************************
%*    LaTex-Vorlage für wissenschaftliche Arbeiten    *
%*    --------------------------------------------    *
%* 		  Autor: Jan Brennenstuhl (@jbspeakr)           *
%*						 www.funkblocka.de                      *
%* 		Version: 0.1								                    *
%* 		  Datum: 07.09.2012							                *
%*     Lizenz: CC BY-SA 3.0                           *
%******************************************************
%* Anmerkung: Basierend auf einer Vorlage von         *
%*            Tino Weinkauf.                          *
%******************************************************
\pdfinfo{                               % Zusatzinformationen in PDF-Datei;
                                        % alle Werte sind optional.
    /Author (Dein Name)
    /CreationDate (D:20130310111111)    % Datum der Erstellung
                                        % (D:JJJJMMTThhmmss)
                                        % JJJJ  Jahr
                                        % MM    Monat
                                        % TT    Tag
                                        % hh    Stunden
                                        % mm    Minuten
                                        % ss    Sekunden
                                        %
                                        % Standard: Das aktuelle Datum
                                        %
    /ModDate (D:20130310111111)         % Datum der letzten Modifikation
    /Creator (TeX \& TXC)               % Standard: "TeX"
    /Producer (pdfTeX)                  % Standard: "pdfTeX" + pdftex version
    /Title () 													% Der Titel deiner Arbeit
    /Subject ()                         % Kurzbeschreibung bspw. "Master-Arbeit, Universität Potsdam"
    /Keywords ()                        % Komma-getrennte Liste von Schlagworten
}
% ---------------------------------------------------------------
% Titelbl�tter und Erkl�rung
\frontmatter
	\pagenumbering{Roman}
		% Titelbl�tter
    %******************************************************
%*    LaTex-Vorlage f�r wissenschaftliche Arbeiten    *
%*    --------------------------------------------    *
%* 		  Autor: Jan Brennenstuhl (@jbspeakr)           *
%*						 www.funkblocka.de                      *
%* 		Version: 0.1								                    *
%* 		  Datum: 07.09.2012							                *
%*     Lizenz: CC BY-SA 3.0                           *
%******************************************************
% Deckblatt
% -----------------------------------------------------
\thispagestyle{empty}
\begin{center}
		
		% Das Logo der Uni (im gfx-Ordner)
    \includegraphics[width=3cm]{Potsdam_logo}\\
    \vspace{.5cm}
    
    % Der Name deiner Uni
    {\Large \sc Universit�t Potsdam}\\
    \vspace{1cm}
    
    % Der Name deines Instituts
    {\Huge \sc Institut f�r Informatik\\[1mm]}
    
    % Der Titel deiner Arbeit (in originaler Sprache)
    \vspace{2cm}
    {\Huge \textbf{Der Titel}}\\
    \vspace*{3mm}
    {\Huge \textbf{deiner Arbeit}}\\
    \vspace*{3mm}
    {\large \textbf{inklusive Untertitel}}\\
    
    % Eine formelle Einleitung
    \vspace{2.0cm}
    {\large Dem Lehrstuhl f�r}\\
    {\large Komplexe Multimediale Anwendungsarchitekturen}\\
    {\large am Institut f�r Informatik der Universit�t Potsdam}\\
    {\large zur Erlangung des akademischen Grades eines}\\
    \vspace{1cm}
    
    % Dein zuk�nftiger akademischer Grad
    {\Large \sc Master of Science}\\
    \vspace{1cm}
    {\large vorgelegte Master-Thesis von}\\
    %\vspace{0.5cm}
    
		% Deine Anrede
    \Large{Herrn/ Frau Dein Name, akadem. Grad}\\
    
    % Dein aktueller Wohnort
    {\large aus Berlin}\\
    \vspace{5cm} % ggf. je nach Zeilenzahl und Schriftgr��e des Titels anpassen
\end{center}

\newpage

% -----------------------------------------------------
% R�ckseite Deckblatt
% -----------------------------------------------------
\thispagestyle{empty}
\cleardoublepage

% -----------------------------------------------------
% Erste Seite (Titelblatt)
% -----------------------------------------------------
\thispagestyle{empty}

\begin{center}
    \vspace{.5cm}
    
    % Der Name deiner Uni
    {\Large \sc Universit�t Potsdam}\\
    \vspace{.5cm}
    
    % Der Name deines Instituts
    {\huge \sc Institut f�r Informatik\\[1mm]}
    \vspace{1cm}

		% Eine kurze formelle Beschreibung
    {\Large Master-Thesis am Lehrstuhl f�r}\\
    {\Large Komplexe Multimediale Anwendungsarchitekturen}\\
    \vspace{1.5cm}
    
    % Der Titel deiner Arbeit (in originaler Sprache)
    {\huge \textbf{Der Titel}}\\
    \vspace*{2mm}
    {\huge \textbf{deiner Arbeit}}\\
    \vspace*{2mm}
    {\large \textbf{inklusive Untertitel}}\\
    
    % Der Titel deiner Arbeit (in englischer bzw. deutscher Entsprechung)
    \vspace{1cm}
    {\huge \textbf{The Title}}\\ 
    \vspace*{2mm}
    {\huge \textbf{of your Thesis}}\\
    \vspace*{2mm}
    {\large \textbf{including Subtitle}}\\
    \vspace{1.5cm}
    
    \parbox{1cm}{
      \begin{large}
        \begin{tabbing}
	       	% Beteiligte Personen (Beschreibung jeweils gender-gerecht!)
	        KandidatIn: \hspace{1.5cm} \=Dein Name, B.Sc.\\[2mm]
	    	  Gutachterin: \>Prof. Dr.-Ing. habil. Deine GutachterIn\\
	    		Gutachter:	 \>Dr. Dein Gutachter\\[2mm]
	    		Abgabedatum: \> 1. Januar 1970\\
        \end{tabbing}
      \end{large}
    }\\
    \vspace{1.5mm}
    
		% Das Logo der Uni (im gfx-Ordner)
    \includegraphics[width=2.4cm]{Potsdam_logo}
\end{center}
 
    \thispagestyle{empty}
    \cleardoublepage
    
    % Erkl�rung (Arbeit selbstst�ndig verfasst)
    %******************************************************
%*    LaTex-Vorlage für wissenschaftliche Arbeiten    *
%*    --------------------------------------------    *
%* 		  Autor: Jan Brennenstuhl (@jbspeakr)           *
%*						 www.funkblocka.de                      *
%* 		Version: 0.1								                    *
%* 		  Datum: 07.09.2012							                *
%*     Lizenz: CC BY-SA 3.0                           *
%******************************************************
\begin{large}

\vspace*{2cm}
\noindent
Ich versichere, dass ich die von mir vorgelegte Arbeit selbstständig verfasst habe, dass ich die verwendeten Quellen, Internet-Quellen und Hilfsmittel vollständig angegeben habe und dass ich die Stellen der Arbeit - einschließlich Tabellen, Karten und Abbildungen -, die anderen Werken oder dem Internet im Wortlaut oder dem Sinn nach entnommen sind, in jedem Fall unter Angabe der Quelle als Entlehnung kenntlich gemacht habe.\vspace{2cm}

\noindent
Berlin, den 1. Januar 1970

\vspace{3cm}

\hspace*{6cm}%
\dotfill\\
\hspace*{7cm}%
\textit{(Unterschrift des Kandidaten)}

\end{large}
  
    \thispagestyle{empty}
    \cleardoublepage
    
    % Zitate
    %******************************************************
%*    LaTex-Vorlage für wissenschaftliche Arbeiten    *
%*    --------------------------------------------    *
%* 		  Autor: Jan Brennenstuhl (@jbspeakr)           *
%*						 www.funkblocka.de                      *
%* 		Version: 0.1								                    *
%* 		  Datum: 07.09.2012							                *
%*     Lizenz: CC BY-SA 3.0                           *
%******************************************************
\vspace*{5cm}

\begin{center}
    \textbf{"`Lorem ipsum dolor sit amet, consetetur sadipscing elitr, sed diam nonumy eirmod tempor invidunt ut labore et"'\\}
\end{center}
 Lorem Ipsum (*1961),\\ ehemaliger Chairman of the Board, Präsident und CEO von Lorem Ipsum

\vspace*{1cm}
\noindent 

    
    % Abstract
    %******************************************************
%*    LaTex-Vorlage f�r wissenschaftliche Arbeiten    *
%*    --------------------------------------------    *
%* 		  Autor: Jan Brennenstuhl (@jbspeakr)           *
%*						 www.funkblocka.de                      *
%* 		Version: 0.1								                    *
%* 		  Datum: 07.09.2012							                *
%*     Lizenz: CC BY-SA 3.0                           *
%******************************************************
% Deutsche Zusammenfassung
\begin{center}
    \textbf{Zusammenfassung}
\end{center}
\vspace*{1cm}
% -----------------------------------------------------
\noindent Lorem ipsum dolor sit amet, consetetur sadipscing elitr, sed diam nonumy eirmod tempor invidunt ut labore et dolore magna aliquyam erat, sed diam voluptua. At vero eos et accusam et justo duo dolores et ea rebum. Stet clita kasd gubergren, no sea takimata sanctus est Lorem ipsum dolor sit amet. Lorem ipsum dolor sit amet, consetetur sadipscing elitr, sed diam nonumy eirmod tempor invidunt ut labore et dolore magna aliquyam erat, sed diam voluptua. At vero eos et accusam et justo duo dolores et ea rebum. Stet clita kasd gubergren, no sea takimata sanctus est Lorem ipsum dolor sit amet. Lorem ipsum dolor sit amet, consetetur sadipscing elitr, sed diam nonumy eirmod tempor invidunt ut labore et dolore magna aliquyam erat, sed diam voluptua. At vero eos et accusam et justo duo dolores et ea rebum. Stet clita kasd gubergren, no sea takimata sanctus est Lorem ipsum dolor sit amet.

% -----------------------------------------------------
% Englische Zusammenfassung
\vspace*{1cm}
\begin{center}
    \textbf{Abstract}
\end{center}
\vspace*{1cm}
% -----------------------------------------------------
\noindent Lorem ipsum dolor sit amet, consetetur sadipscing elitr, sed diam nonumy eirmod tempor invidunt ut labore et dolore magna aliquyam erat, sed diam voluptua. At vero eos et accusam et justo duo dolores et ea rebum. Stet clita kasd gubergren, no sea takimata sanctus est Lorem ipsum dolor sit amet. Lorem ipsum dolor sit amet, consetetur sadipscing elitr, sed diam nonumy eirmod tempor invidunt ut labore et dolore magna aliquyam erat, sed diam voluptua. At vero eos et accusam et justo duo dolores et ea rebum. Stet clita kasd gubergren, no sea takimata sanctus est Lorem ipsum dolor sit amet. Lorem ipsum dolor sit amet, consetetur sadipscing elitr, sed diam nonumy eirmod tempor invidunt ut labore et dolore magna aliquyam erat, sed diam voluptua. At vero eos et accusam et justo duo dolores et ea rebum. Stet clita kasd gubergren, no sea takimata sanctus est Lorem ipsum dolor sit amet.
    
    % Inhaltsverzeichnis
    \tableofcontents 
% ---------------------------------------------------------------
% Hauptteil - Arabische Seitennummerierung
\mainmatter % die eigentliche Arbeit
		\pagenumbering{arabic}
		%******************************************************
%*    LaTex-Vorlage für wissenschaftliche Arbeiten    *
%*    --------------------------------------------    *
%*      Autor: Jan Brennenstuhl (@jbspeakr)           *
%*             www.funkblocka.de                      *
%*            Lizenz: CC BY-SA 3.0                    *
%******************************************************
\chapter{Einleitung}
\label{chap:einleitung}
Lorem ipsum dolor \gls{ID} amet, consetetur sadipscing elitr, sed diam nonumy eirmod tempor invidunt ut labore et dolore magna aliquyam erat, sed diam voluptua. At vero eos et accusam et justo duo dolores et ea rebum. Stet clita kasd gubergren, no sea takimata sanctus est Lorem ipsum dolor sit amet. Lorem ipsum dolor sit amet, consetetur sadipscing elitr, sed diam nonumy eirmod tempor invidunt ut labore et dolore magna aliquyam erat, sed diam voluptua. At vero eos et accusam et justo duo dolores et ea rebum. Stet clita kasd gubergren, no sea takimata sanctus est Lorem ipsum dolor sit amet. Lorem ipsum dolor sit amet, consetetur sadipscing elitr, sed diam nonumy eirmod tempor invidunt ut labore et dolore magna aliquyam erat, sed diam voluptua. At vero eos et accusam et justo duo dolores et ea rebum. Stet clita kasd gubergren, no sea takimata sanctus est Lorem ipsum dolor sit amet \citeweb[][]{exmpl:web}.
% -----------------------------------------------------
\section{Motivation}
Lorem ipsum dolor sit amet\footnote{Lorem ipsum dolor sit amet, consetetur \gls{glos:PKI} sadipscing elitr.}, consetetur sadipscing elitr, sed diam nonumy eirmod tempor invidunt ut labore et dolore magna aliquyam erat, sed diam voluptua. At vero eos et accusam et justo duo dolores et ea rebum. Stet clita kasd gubergren, no sea takimata sanctus est Lorem ipsum dolor sit amet. Lorem ipsum dolor sit amet, consetetur sadipscing elitr, sed diam nonumy eirmod tempor invidunt ut labore et dolore magna aliquyam erat, sed diam voluptua. At vero eos et accusam et justo duo dolores et ea rebum. Stet clita kasd gubergren, no sea takimata sanctus est Lorem ipsum dolor sit amet. Lorem ipsum dolor sit amet, consetetur sadipscing elitr, sed diam nonumy eirmod tempor invidunt ut labore et dolore magna aliquyam erat, sed diam voluptua. At vero eos et accusam et justo duo dolores et ea rebum. Stet clita kasd gubergren, no sea takimata sanctus est Lorem ipsum dolor sit amet \citeliterature[vgl.][S. 28ff]{exmpl:buch}.
% -----------------------------------------------------
\section{Anwendungsgebiet}
Lorem ipsum dolor sit amet, consetetur sadipscing elitr, "`sed diam nonumy eirmod tempor invidunt ut labore et dolore magna aliquyam erat, sed diam voluptua"' \citeliterature[][S. 12]{exmpl:buch2}. At vero eos et accusam et justo duo dolores et ea rebum. Stet clita kasd gubergren, no sea takimata sanctus est Lorem ipsum dolor sit amet. Lorem ipsum dolor sit amet, consetetur sadipscing elitr, sed diam nonumy eirmod tempor invidunt ut labore et dolore magna aliquyam erat, sed diam voluptua. At vero eos et accusam et justo duo dolores et ea rebum. Stet clita kasd gubergren, no sea takimata sanctus est Lorem ipsum dolor sit amet. Lorem ipsum dolor sit amet, consetetur sadipscing elitr, sed diam nonumy eirmod tempor invidunt ut labore et dolore magna aliquyam erat, sed diam voluptua. At vero eos et accusam et justo duo dolores et ea rebum. Stet clita kasd gubergren, no sea takimata sanctus est Lorem ipsum dolor sit amet.
% -----------------------------------------------------
\section{Zielstellung}
Lorem ipsum dolor sit amet, \glspl{PKI} consetetur sadipscing elitr, sed diam nonumy eirmod tempor invidunt ut labore et dolore magna aliquyam erat, sed diam voluptua. At vero eos et accusam et justo duo dolores et ea rebum. Stet clita kasd gubergren, no sea takimata sanctus est Lorem ipsum dolor sit amet. Lorem ipsum dolor sit amet, consetetur sadipscing elitr, sed diam nonumy eirmod tempor invidunt ut labore et dolore magna aliquyam erat, sed diam voluptua. At vero eos et accusam et justo duo dolores et ea rebum. Stet clita kasd gubergren, no sea takimata sanctus est Lorem ipsum dolor sit amet. Lorem ipsum dolor sit amet, consetetur sadipscing elitr, sed diam nonumy eirmod tempor invidunt ut labore et dolore magna aliquyam erat, sed diam voluptua. At vero eos et accusam et justo duo dolores et ea rebum. Stet clita kasd gubergren, no sea takimata sanctus est Lorem ipsum dolor sit amet.
% -----------------------------------------------------
\section{Aufbau}
Lorem ipsum dolor sit amet, consetetur sadipscing elitr, sed diam nonumy eirmod tempor invidunt ut labore et dolore magna aliquyam erat, sed diam voluptua. At vero eos et accusam et justo duo dolores et ea rebum. Stet clita kasd gubergren, no sea takimata sanctus est Lorem ipsum dolor sit amet. Lorem ipsum dolor sit amet, consetetur sadipscing elitr, sed diam nonumy eirmod tempor invidunt ut labore et dolore magna aliquyam erat, sed diam voluptua. At vero eos et accusam et justo duo dolores et ea rebum. Stet clita kasd gubergren, no sea takimata sanctus est Lorem ipsum dolor sit amet. Lorem ipsum dolor sit amet, consetetur sadipscing elitr, sed diam nonumy eirmod tempor invidunt ut labore et dolore magna aliquyam erat, sed diam voluptua. At vero eos et accusam et justo duo dolores et ea rebum. Stet clita kasd gubergren, no sea takimata sanctus est Lorem ipsum dolor sit amet.
% -----------------------------------------------------
\endinput 
		\input{2-Second}
    \input{3-Third}
    %******************************************************
%*    LaTex-Vorlage für wissenschaftliche Arbeiten    *
%*    --------------------------------------------    *
%*      Autor: Jan Brennenstuhl (@jbspeakr)           *
%*             www.funkblocka.de                      *
%*            Lizenz: CC BY-SA 3.0                    *
%******************************************************
\chapter{Kapitel 4}
\label{chap:four}
Lorem ipsum dolor sit amet, consetetur sadipscing elitr, sed diam nonumy eirmod tempor invidunt ut labore et dolore magna aliquyam erat, sed diam voluptua. At vero eos et accusam et justo duo dolores et ea rebum. Stet clita kasd gubergren, no sea takimata sanctus est Lorem ipsum dolor sit amet. Lorem ipsum dolor sit amet, consetetur sadipscing elitr, sed diam nonumy eirmod tempor invidunt ut labore et dolore magna aliquyam erat, sed diam voluptua. At vero eos et accusam et justo duo dolores et ea rebum. Stet clita kasd gubergren, no sea takimata sanctus est Lorem ipsum dolor sit amet. Lorem ipsum dolor sit amet, consetetur sadipscing elitr, sed diam nonumy eirmod tempor invidunt ut labore et dolore magna aliquyam erat, sed diam voluptua. At vero eos et accusam et justo duo dolores et ea rebum. Stet clita kasd gubergren, no sea takimata sanctus est Lorem ipsum dolor sit amet.
% -----------------------------------------------------
\endinput 
    \input{5-Fifth}
    \input{6-Sixth}
    %******************************************************
%*    LaTex-Vorlage für wissenschaftliche Arbeiten    *
%*    --------------------------------------------    *
%*      Autor: Jan Brennenstuhl (@jbspeakr)           *
%*             www.funkblocka.de                      *
%*            Lizenz: CC BY-SA 3.0                    *
%******************************************************
\chapter{Kapitel 7}
\label{chap:seven}
Lorem ipsum dolor sit amet, consetetur sadipscing elitr, sed diam nonumy eirmod tempor invidunt ut labore et dolore magna aliquyam erat, sed diam voluptua. At vero eos et accusam et justo duo dolores et ea rebum. Stet clita kasd gubergren, no sea takimata sanctus est Lorem ipsum dolor sit amet. Lorem ipsum dolor sit amet, consetetur sadipscing elitr, sed diam nonumy eirmod tempor invidunt ut labore et dolore magna aliquyam erat, sed diam voluptua. At vero eos et accusam et justo duo dolores et ea rebum. Stet clita kasd gubergren, no sea takimata sanctus est Lorem ipsum dolor sit amet. Lorem ipsum dolor sit amet, consetetur sadipscing elitr, sed diam nonumy eirmod tempor invidunt ut labore et dolore magna aliquyam erat, sed diam voluptua. At vero eos et accusam et justo duo dolores et ea rebum. Stet clita kasd gubergren, no sea takimata sanctus est Lorem ipsum dolor sit amet.
% -----------------------------------------------------
\section{Zusammenfassung und erzielte Ergebnisse}
Lorem ipsum dolor sit amet, consetetur sadipscing elitr, sed diam nonumy eirmod tempor invidunt ut labore et dolore magna aliquyam erat, sed diam voluptua. At vero eos et accusam et justo duo dolores et ea rebum. Stet clita kasd gubergren, no sea takimata sanctus est Lorem ipsum dolor sit amet. Lorem ipsum dolor sit amet, consetetur sadipscing elitr, sed diam nonumy eirmod tempor invidunt ut labore et dolore magna aliquyam erat, sed diam voluptua. At vero eos et accusam et justo duo dolores et ea rebum. Stet clita kasd gubergren, no sea takimata sanctus est Lorem ipsum dolor sit amet. Lorem ipsum dolor sit amet, consetetur sadipscing elitr, sed diam nonumy eirmod tempor invidunt ut labore et dolore magna aliquyam erat, sed diam voluptua. At vero eos et accusam et justo duo dolores et ea rebum. Stet clita kasd gubergren, no sea takimata sanctus est Lorem ipsum dolor sit amet.
% -----------------------------------------------------
\section{Ausblick und weiterführende Fragestellungen}
Lorem ipsum dolor sit amet, consetetur sadipscing elitr, sed diam nonumy eirmod tempor invidunt ut labore et dolore magna aliquyam erat, sed diam voluptua. At vero eos et accusam et justo duo dolores et ea rebum. Stet clita kasd gubergren, no sea takimata sanctus est Lorem ipsum dolor sit amet. Lorem ipsum dolor sit amet, consetetur sadipscing elitr, sed diam nonumy eirmod tempor invidunt ut labore et dolore magna aliquyam erat, sed diam voluptua. At vero eos et accusam et justo duo dolores et ea rebum. Stet clita kasd gubergren, no sea takimata sanctus est Lorem ipsum dolor sit amet. Lorem ipsum dolor sit amet, consetetur sadipscing elitr, sed diam nonumy eirmod tempor invidunt ut labore et dolore magna aliquyam erat, sed diam voluptua. At vero eos et accusam et justo duo dolores et ea rebum. Stet clita kasd gubergren, no sea takimata sanctus est Lorem ipsum dolor sit amet.
% -----------------------------------------------------
\endinput 
% ---------------------------------------------------------------
% Schlussteil - R�mische Seitennummerierung
\backmatter
\pagenumbering{Roman}
\setcounter{page}{11}
% ---------------------------------------------------------------
% Quellenverzeichnisse
\bibliographystylebuch{natdin}
\bibliographybuch{buch}
\bibliographystyleweb{natdin}
\bibliographyweb{web}
% ---------------------------------------------------------------
% Glossar
\renewcommand{\glossaryname}{Glossar}
\printglossary
\addcontentsline{toc}{chapter}{Glossar}
% ---------------------------------------------------------------
% Anhang
\appendix
%\includepdf[pages=-,fitpaper=true,addtotoc={1,section,0,Diagramm: Vergabe,fig:Vergabedia}]{Pfad-zu-einem-pdf-dokument.pdf}
% ---------------------------------------------------------------
\end{document}